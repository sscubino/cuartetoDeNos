\documentclass{article}
\usepackage{ifthen}
\usepackage{amssymb}
\usepackage{multicol}
\usepackage{graphicx}
\usepackage[absolute]{textpos}

%\usepackage{xspace,rotating,calligra,dsfont,ifthen}
\usepackage{xspace,rotating,dsfont,ifthen}
\usepackage[spanish,activeacute]{babel}
\usepackage[utf8]{inputenc}
\usepackage{pgfpages}
\usepackage{pgf,pgfarrows,pgfnodes,pgfautomata,pgfheaps,xspace,dsfont}
\usepackage{listings}
\usepackage{multicol}


\makeatletter

\@ifclassloaded{beamer}{%
  \newcommand{\tocarEspacios}{%
    \addtolength{\leftskip}{4em}%
    \addtolength{\parindent}{-3em}%
  }%
}
{%
  \usepackage[top=1cm,bottom=2cm,left=1cm,right=1cm]{geometry}%
  \usepackage{color}%
  \newcommand{\tocarEspacios}{%
    \addtolength{\leftskip}{5em}%
    \addtolength{\parindent}{-3em}%
  }%
}

\newcommand{\encabezadoDeProc}[4]{%
  % Ponemos la palabrita problema en tt
%  \noindent%
  {\normalfont\bfseries\ttfamily proc}%
  % Ponemos el nombre del problema
  \ %
  {\normalfont\ttfamily #2}%
  \
  % Ponemos los parametros
  (#3)%
  \ifthenelse{\equal{#4}{}}{}{%
  \ =\ %
  % Ponemos el nombre del resultado
  {\normalfont\ttfamily #1}%
  % Por ultimo, va el tipo del resultado
  \ : #4}
}

\newcommand{\encabezadoDeTipo}[2]{%
  % Ponemos la palabrita tipo en tt
  {\normalfont\bfseries\ttfamily tipo}%
  % Ponemos el nombre del tipo
  \ %
  {\normalfont\ttfamily #2}%
  \ifthenelse{\equal{#1}{}}{}{$\langle$#1$\rangle$}
}

% Primero definiciones de cosas al estilo title, author, date

\def\materia#1{\gdef\@materia{#1}}
\def\@materia{No especifi\'o la materia}
\def\lamateria{\@materia}

\def\cuatrimestre#1{\gdef\@cuatrimestre{#1}}
\def\@cuatrimestre{No especifi\'o el cuatrimestre}
\def\elcuatrimestre{\@cuatrimestre}

\def\anio#1{\gdef\@anio{#1}}
\def\@anio{No especifi\'o el anio}
\def\elanio{\@anio}

\def\fecha#1{\gdef\@fecha{#1}}
\def\@fecha{\today}
\def\lafecha{\@fecha}

\def\nombre#1{\gdef\@nombre{#1}}
\def\@nombre{No especific'o el nombre}
\def\elnombre{\@nombre}

\def\practicas#1{\gdef\@practica{#1}}
\def\@practica{No especifi\'o el n\'umero de pr\'actica}
\def\lapractica{\@practica}


% Esta macro convierte el numero de cuatrimestre a palabras
\newcommand{\cuatrimestreLindo}{
  \ifthenelse{\equal{\elcuatrimestre}{1}}
  {Primer cuatrimestre}
  {\ifthenelse{\equal{\elcuatrimestre}{2}}
  {Segundo cuatrimestre}
  {Verano}}
}


\newcommand{\depto}{{UBA -- Facultad de Ciencias Exactas y Naturales --
      Departamento de Computaci\'on}}

\newcommand{\titulopractica}{
  \centerline{\depto}
  \vspace{1ex}
  \centerline{{\Large\lamateria}}
  \vspace{0.5ex}
  \centerline{\cuatrimestreLindo de \elanio}
  \vspace{2ex}
  \centerline{{\huge Pr\'actica \lapractica -- \elnombre}}
  \vspace{5ex}
  \arreglarincisos
  \newcounter{ejercicio}
  \newenvironment{ejercicio}{\stepcounter{ejercicio}\textbf{Ejercicio
      \theejercicio}%
    \renewcommand\@currentlabel{\theejercicio}%
  }{\vspace{0.2cm}}
}


\newcommand{\titulotp}{
  \centerline{\depto}
  \vspace{1ex}
  \centerline{{\Large\lamateria}}
  \vspace{0.5ex}
  \centerline{\cuatrimestreLindo de \elanio}
  \vspace{0.5ex}
  \centerline{\lafecha}
  \vspace{2ex}
  \centerline{{\huge\elnombre}}
  \vspace{5ex}
}


%practicas
\newcommand{\practica}[2]{%
    \title{Pr\'actica #1 \\ #2}
    \author{Algoritmos y Estructuras de Datos I}
    \date{Primer Cuatrimestre 2018}

    \maketitlepractica{#1}{#2}
}

\newcommand \maketitlepractica[2] {%
\begin{center}
\begin{tabular}{r cr}
 \begin{tabular}{c}
{\large\bf\textsf{\ Algoritmos y Estructuras de Datos I\ }}\\
Primer Cuatrimestre 2018\\
\title{\normalsize Gu\'ia Pr\'actica #1 \\ \textbf{#2}}\\
\@title
\end{tabular} &
\begin{tabular}{@{} p{1.6cm} @{}}
\includegraphics[width=1.6cm]{logodpt.jpg}
\end{tabular} &
\begin{tabular}{l @{}}
 \emph{Departamento de Computaci\'on} \\
 \emph{Facultad de Ciencias Exactas y Naturales} \\
 \emph{Universidad de Buenos Aires} \\
\end{tabular}
\end{tabular}
\end{center}

\bigskip
}


% Simbolos varios

\newcommand{\ent}{\ensuremath{\mathds{Z}}}
\newcommand{\float}{\ensuremath{\mathds{R}}}
\newcommand{\bool}{\ensuremath{\mathsf{Bool}}}
\newcommand{\True}{\ensuremath{\mathrm{true}}}
\newcommand{\False}{\ensuremath{\mathrm{false}}}
\newcommand{\Then}{\ensuremath{\rightarrow}}
\newcommand{\Iff}{\ensuremath{\leftrightarrow}}
\newcommand{\implica}{\ensuremath{\longrightarrow}}
\newcommand{\IfThenElse}[3]{\ensuremath{\mathsf{if}\ #1\ \mathsf{then}\ #2\ \mathsf{else}\ #3\ \mathsf{fi}}}
\newcommand{\In}{\textsf{in }}
\newcommand{\Out}{\textsf{out }}
\newcommand{\Inout}{\textsf{inout }}
\newcommand{\yLuego}{\land _L}
\newcommand{\oLuego}{\lor _L}
\newcommand{\implicaLuego}{\implica _L}
\newcommand{\existe}[3]{\ensuremath{(\exists #1:\ent) \ #2 \leq #1 < #3 \ }}
\newcommand{\paraTodo}[3]{\ensuremath{(\forall #1:\ent) \ #2 \leq #1 < #3 \ }}

% Símbolo para marcar los ejercicios importantes (estrellita)
\newcommand\importante{\raisebox{0.5pt}{\ensuremath{\bigstar}}}


\newcommand{\rango}[2]{[#1\twodots#2]}
\newcommand{\comp}[2]{[\,#1\,|\,#2\,]}

\newcommand{\rangoac}[2]{(#1\twodots#2]}
\newcommand{\rangoca}[2]{[#1\twodots#2)}
\newcommand{\rangoaa}[2]{(#1\twodots#2)}

%ejercicios
\newtheorem{exercise}{Ejercicio}
\newenvironment{ejercicio}[1][]{\begin{exercise}#1\rm}{\end{exercise} \vspace{0.2cm}}
\newenvironment{items}{\begin{enumerate}[a)]}{\end{enumerate}}
\newenvironment{subitems}{\begin{enumerate}[i)]}{\end{enumerate}}
\newcommand{\sugerencia}[1]{\noindent \textbf{Sugerencia:} #1}

\lstnewenvironment{code}{
    \lstset{% general command to set parameter(s)
        language=C++, basicstyle=\small\ttfamily, keywordstyle=\slshape,
        emph=[1]{tipo,usa}, emphstyle={[1]\sffamily\bfseries},
        morekeywords={tint,forn,forsn},
        basewidth={0.47em,0.40em},
        columns=fixed, fontadjust, resetmargins, xrightmargin=5pt, xleftmargin=15pt,
        flexiblecolumns=false, tabsize=2, breaklines, breakatwhitespace=false, extendedchars=true,
        numbers=left, numberstyle=\tiny, stepnumber=1, numbersep=9pt,
        frame=l, framesep=3pt,
    }
   \csname lst@SetFirstLabel\endcsname}
  {\csname lst@SaveFirstLabel\endcsname}


%tipos basicos
\newcommand{\rea}{\ensuremath{\mathsf{Float}}}
\newcommand{\cha}{\ensuremath{\mathsf{Char}}}
\newcommand{\str}{\ensuremath{\mathsf{String}}}

\newcommand{\mcd}{\mathrm{mcd}}
\newcommand{\prm}[1]{\ensuremath{\mathsf{prm}(#1)}}
\newcommand{\sgd}[1]{\ensuremath{\mathsf{sgd}(#1)}}

\newcommand{\tuple}[2]{\ensuremath{#1 \times #2}}

%listas
\newcommand{\TLista}[1]{\ensuremath{seq \langle #1\rangle}}
\newcommand{\lvacia}{\ensuremath{[\ ]}}
\newcommand{\lv}{\ensuremath{[\ ]}}
\newcommand{\longitud}[1]{\ensuremath{|#1|}}
\newcommand{\cons}[1]{\ensuremath{\mathsf{addFirst}}(#1)}
\newcommand{\indice}[1]{\ensuremath{\mathsf{indice}}(#1)}
\newcommand{\conc}[1]{\ensuremath{\mathsf{concat}}(#1)}
\newcommand{\cab}[1]{\ensuremath{\mathsf{head}}(#1)}
\newcommand{\cola}[1]{\ensuremath{\mathsf{tail}}(#1)}
\newcommand{\sub}[1]{\ensuremath{\mathsf{subseq}}(#1)}
\newcommand{\en}[1]{\ensuremath{\mathsf{en}}(#1)}
\newcommand{\cuenta}[2]{\mathsf{cuenta}\ensuremath{(#1, #2)}}
\newcommand{\suma}[1]{\mathsf{suma}(#1)}
\newcommand{\twodots}{\ensuremath{\mathrm{..}}}
\newcommand{\masmas}{\ensuremath{++}}
\newcommand{\matriz}[1]{\TLista{\TLista{#1}}}

% Acumulador
\newcommand{\acum}[1]{\ensuremath{\mathsf{acum}}(#1)}
\newcommand{\acumselec}[3]{\ensuremath{\mathrm{acum}(#1 |  #2, #3)}}

% \selector{variable}{dominio}
\newcommand{\selector}[2]{#1~\ensuremath{\leftarrow}~#2}
\newcommand{\selec}{\ensuremath{\leftarrow}}

\newcommand{\pred}[3]{%
    {\normalfont\bfseries\ttfamily pred }%
    {\normalfont\ttfamily #1}%
    \ifthenelse{\equal{#2}{}}{}{\ (#2) }%
    \{\ensuremath{#3}\}%
    {\normalfont\bfseries\,\par}%
  }

\newenvironment{proc}[4][res]{%
  % El parametro 1 (opcional) es el nombre del resultado
  % El parametro 2 es el nombre del problema
  % El parametro 3 son los parametros
  % El parametro 4 es el tipo del resultado
  % Preambulo del ambiente problema
  % Tenemos que definir los comandos requiere, asegura, modifica y aux
  \newcommand{\pre}[2][]{%
    {\normalfont\bfseries\ttfamily Pre}%
    \ifthenelse{\equal{##1}{}}{}{\ {\normalfont\ttfamily ##1} :}\ %
    \{\ensuremath{##2}\}%
    {\normalfont\bfseries\,\par}%
  }
  \newcommand{\post}[2][]{%
    {\normalfont\bfseries\ttfamily Post}%
    \ifthenelse{\equal{##1}{}}{}{\ {\normalfont\ttfamily ##1} :}\
    \{\ensuremath{##2}\}%
    {\normalfont\bfseries\,\par}%
  }
  \renewcommand{\aux}[4]{%
    {\normalfont\bfseries\ttfamily fun\ }%
    {\normalfont\ttfamily ##1}%
    \ifthenelse{\equal{##2}{}}{}{\ (##2)}\ : ##3\, = \ensuremath{##4}%
    {\normalfont\bfseries\,;\par}%
  }
  \newcommand{\res}{#1}
  \vspace{1ex}
  \noindent
  \encabezadoDeProc{#1}{#2}{#3}{#4}
  % Abrimos la llave
  \{\par%
  \tocarEspacios
}
% Ahora viene el cierre del ambiente problema
{
  % Cerramos la llave
  \noindent\}
  \vspace{1ex}
}


  \newcommand{\aux}[4]{%
    {\normalfont\bfseries\ttfamily fun\ }%
    {\normalfont\ttfamily #1}%
    \ifthenelse{\equal{#2}{}}{}{\ (#2)}\ : #3\, = \ensuremath{#4}%
    {\normalfont\bfseries\,;\par}%
  }


% \newcommand{\pre}[1]{\textsf{pre}\ensuremath{(#1)}}

\newcommand{\procnom}[1]{\textsf{#1}}
\newcommand{\procil}[3]{\textsf{proc #1}\ensuremath{(#2) = #3}}
\newcommand{\procilsinres}[2]{\textsf{proc #1}\ensuremath{(#2)}}
\newcommand{\preil}[2]{\textsf{Pre #1: }\ensuremath{#2}}
\newcommand{\postil}[2]{\textsf{Post #1: }\ensuremath{#2}}
\newcommand{\auxil}[2]{\textsf{fun }\ensuremath{#1 = #2}}
\newcommand{\auxilc}[4]{\textsf{fun }\ensuremath{#1( #2 ): #3 = #4}}
\newcommand{\auxnom}[1]{\textsf{fun }\ensuremath{#1}}
\newcommand{\auxpred}[3]{\textsf{pred }\ensuremath{#1( #2 ) \{ #3 \}}}

\newcommand{\comentario}[1]{{/*\ #1\ */}}

\newcommand{\nom}[1]{\ensuremath{\mathsf{#1}}}


% En las practicas/parciales usamos numeros arabigos para los ejercicios.
% Aca cambiamos los enumerate comunes para que usen letras y numeros
% romanos
\newcommand{\arreglarincisos}{%
  \renewcommand{\theenumi}{\alph{enumi}}
  \renewcommand{\theenumii}{\roman{enumii}}
  \renewcommand{\labelenumi}{\theenumi)}
  \renewcommand{\labelenumii}{\theenumii)}
}



%%%%%%%%%%%%%%%%%%%%%%%%%%%%%% PARCIAL %%%%%%%%%%%%%%%%%%%%%%%%
\let\@xa\expandafter
\newcommand{\tituloparcial}{\centerline{\depto -- \lamateria}
  \centerline{\elnombre -- \lafecha}%
  \setlength{\TPHorizModule}{10mm} % Fija las unidades de textpos
  \setlength{\TPVertModule}{\TPHorizModule} % Fija las unidades de
                                % textpos
  \arreglarincisos
  \newcounter{total}% Este contador va a guardar cuantos incisos hay
                    % en el parcial. Si un ejercicio no tiene incisos,
                    % cuenta como un inciso.
  \newcounter{contgrilla} % Para hacer ciclos
  \newcounter{columnainicial} % Se van a usar para los cline cuando un
  \newcounter{columnafinal}   % ejercicio tenga incisos.
  \newcommand{\primerafila}{}
  \newcommand{\segundafila}{}
  \newcommand{\rayitas}{} % Esto va a guardar los \cline de los
                          % ejercicios con incisos, asi queda mas bonito
  \newcommand{\anchodegrilla}{20} % Es para textpos
  \newcommand{\izquierda}{7} % Estos dos le dicen a textpos donde colocar
  \newcommand{\abajo}{2}     % la grilla
  \newcommand{\anchodecasilla}{0.4cm}
  \setcounter{columnainicial}{1}
  \setcounter{total}{0}
  \newcounter{ejercicio}
  \setcounter{ejercicio}{0}
  \renewenvironment{ejercicio}[1]
  {%
    \stepcounter{ejercicio}\textbf{\noindent Ejercicio \theejercicio. [##1
      puntos]}% Formato
    \renewcommand\@currentlabel{\theejercicio}% Esto es para las
                                % referencias
    \newcommand{\invariante}[2]{%
      {\normalfont\bfseries\ttfamily invariante}%
      \ ####1\hspace{1em}####2%
    }%
    \newcommand{\Proc}[5][result]{
      \encabezadoDeProc{####1}{####2}{####3}{####4}\hspace{1em}####5}%
  }% Aca se termina el principio del ejercicio
  {% Ahora viene el final
    % Esto suma la cantidad de incisos o 1 si no hubo ninguno
    \ifthenelse{\equal{\value{enumi}}{0}}
    {\addtocounter{total}{1}}
    {\addtocounter{total}{\value{enumi}}}
    \ifthenelse{\equal{\value{ejercicio}}{1}}{}
    {
      \g@addto@macro\primerafila{&} % Si no estoy en el primer ej.
      \g@addto@macro\segundafila{&}
    }
    \ifthenelse{\equal{\value{enumi}}{0}}
    {% No tiene incisos
      \g@addto@macro\primerafila{\multicolumn{1}{|c|}}
      \bgroup% avoid overwriting somebody else's value of \tmp@a
      \protected@edef\tmp@a{\theejercicio}% expand as far as we can
      \@xa\g@addto@macro\@xa\primerafila\@xa{\tmp@a}%
      \egroup% restore old value of \tmp@a, effect of \g@addto.. is

      \stepcounter{columnainicial}
    }
    {% Tiene incisos
      % Primero ponemos el encabezado
      \g@addto@macro\primerafila{\multicolumn}% Ahora el numero de items
      \bgroup% avoid overwriting somebody else's value of \tmp@a
      \protected@edef\tmp@a{\arabic{enumi}}% expand as far as we can
      \@xa\g@addto@macro\@xa\primerafila\@xa{\tmp@a}%
      \egroup% restore old value of \tmp@a, effect of \g@addto.. is
      % global
      % Ahora el formato
      \g@addto@macro\primerafila{{|c|}}%
      % Ahora el numero de ejercicio
      \bgroup% avoid overwriting somebody else's value of \tmp@a
      \protected@edef\tmp@a{\theejercicio}% expand as far as we can
      \@xa\g@addto@macro\@xa\primerafila\@xa{\tmp@a}%
      \egroup% restore old value of \tmp@a, effect of \g@addto.. is
      % global
      % Ahora armamos la segunda fila
      \g@addto@macro\segundafila{\multicolumn{1}{|c|}{a}}%
      \setcounter{contgrilla}{1}
      \whiledo{\value{contgrilla}<\value{enumi}}
      {%
        \stepcounter{contgrilla}
        \g@addto@macro\segundafila{&\multicolumn{1}{|c|}}
        \bgroup% avoid overwriting somebody else's value of \tmp@a
        \protected@edef\tmp@a{\alph{contgrilla}}% expand as far as we can
        \@xa\g@addto@macro\@xa\segundafila\@xa{\tmp@a}%
        \egroup% restore old value of \tmp@a, effect of \g@addto.. is
        % global
      }
      % Ahora armo las rayitas
      \setcounter{columnafinal}{\value{columnainicial}}
      \addtocounter{columnafinal}{-1}
      \addtocounter{columnafinal}{\value{enumi}}
      \bgroup% avoid overwriting somebody else's value of \tmp@a
      \protected@edef\tmp@a{\noexpand\cline{%
          \thecolumnainicial-\thecolumnafinal}}%
      \@xa\g@addto@macro\@xa\rayitas\@xa{\tmp@a}%
      \egroup% restore old value of \tmp@a, effect of \g@addto.. is
      \setcounter{columnainicial}{\value{columnafinal}}
      \stepcounter{columnainicial}
    }
    \setcounter{enumi}{0}%
    \vspace{0.2cm}%
  }%
  \newcommand{\tercerafila}{}
  \newcommand{\armartercerafila}{
    \setcounter{contgrilla}{1}
    \whiledo{\value{contgrilla}<\value{total}}
    {\stepcounter{contgrilla}\g@addto@macro\tercerafila{&}}
  }
  \newcommand{\grilla}{%
    \g@addto@macro\primerafila{&\textbf{TOTAL}}
    \g@addto@macro\segundafila{&}
    \g@addto@macro\tercerafila{&}
    \armartercerafila
    \ifthenelse{\equal{\value{total}}{\value{ejercicio}}}
    {% No hubo incisos
      \begin{textblock}{\anchodegrilla}(\izquierda,\abajo)
        \begin{tabular}{|*{\value{total}}{p{\anchodecasilla}|}c|}
          \hline
          \primerafila\\
          \hline
          \tercerafila\\
          \tercerafila\\
          \hline
        \end{tabular}
      \end{textblock}
    }
    {% Hubo incisos
      \begin{textblock}{\anchodegrilla}(\izquierda,\abajo)
        \begin{tabular}{|*{\value{total}}{p{\anchodecasilla}|}c|}
          \hline
          \primerafila\\
          \rayitas
          \segundafila\\
          \hline
          \tercerafila\\
          \tercerafila\\
          \hline
        \end{tabular}
      \end{textblock}
    }
  }%
  \vspace{0.4cm}
  \textbf{Nro. de orden:}

  \textbf{LU:}

  \textbf{Apellidos:}

  \textbf{Nombres:}
  \vspace{0.5cm}
}



% AMBIENTE CONSIGNAS
% Se usa en el TP para ir agregando las cosas que tienen que resolver
% los alumnos.
% Dentro del ambiente hay que usar \item para cada consigna

\newcounter{consigna}
\setcounter{consigna}{0}

\newenvironment{consignas}{%
  \newcommand{\consigna}{\stepcounter{consigna}\textbf{\theconsigna.}}%
  \renewcommand{\ejercicio}[1]{\item ##1 }
  \renewcommand{\proc}[5][result]{\item
    \encabezadoDeProc{##1}{##2}{##3}{##4}\hspace{1em}##5}%
  \newcommand{\invariante}[2]{\item%
    {\normalfont\bfseries\ttfamily invariante}%
    \ ##1\hspace{1em}##2%
  }
  \renewcommand{\aux}[4]{\item%
    {\normalfont\bfseries\ttfamily aux\ }%
    {\normalfont\ttfamily ##1}%
    \ifthenelse{\equal{##2}{}}{}{\ (##2)}\ : ##3 \hspace{1em}##4%
  }
  % Comienza la lista de consignas
  \begin{list}{\consigna}{%
      \setlength{\itemsep}{0.5em}%
      \setlength{\parsep}{0cm}%
    }
}%
{\end{list}}



% para decidir si usar && o ^
\newcommand{\y}[0]{\ensuremath{\land}}

% macros de correctitud
\newcommand{\semanticComment}[2]{#1 \ensuremath{#2};}
\newcommand{\namedSemanticComment}[3]{#1 #2: \ensuremath{#3};}


\newcommand{\local}[1]{\semanticComment{local}{#1}}

\newcommand{\vale}[1]{\semanticComment{vale}{#1}}
\newcommand{\valeN}[2]{\namedSemanticComment{vale}{#1}{#2}}
\newcommand{\impl}[1]{\semanticComment{implica}{#1}}
\newcommand{\implN}[2]{\namedSemanticComment{implica}{#1}{#2}}
\newcommand{\estado}[1]{\semanticComment{estado}{#1}}

\newcommand{\invarianteCN}[2]{\namedSemanticComment{invariante}{#1}{#2}}
\newcommand{\invarianteC}[1]{\semanticComment{invariante}{#1}}
\newcommand{\varianteCN}[2]{\namedSemanticComment{variante}{#1}{#2}}
\newcommand{\varianteC}[1]{\semanticComment{variante}{#1}}

\usepackage{caratula}
\usepackage{scrextend}


\begin{document}

%Carátula
\titulo{TP 1 - Reuniones Remotas}
\subtitulo{Grupo 3}
\fecha{22 de Mayo de 2020}
\materia{Algoritmos y Estructuras de Datos 1}
\integrante{González Narvarte, Francisco}{519/15}{francisco13\_95@live.com}
\integrante{Giménez, Iván Manuel}{374/18}{ivangimenez8727@gmail.com}
\integrante{Demare, Matías Nicolás}{762/19}{matiasdemare@gmail.com}
\integrante{Cubino, Santiago}{829/19}{sscubino@gmail.com}
\maketitle

%Creación de índice
\tableofcontents
\newpage

% End carátula




\addcontentsline{toc}{section}{Ejercicio 1}
\section*{Ejercicio 1}
\vspace{1cm}
\begin{addmargin}[4em]{0em}
	\begin{proc}{esSeñal}{\In s :\TLista{\ent}, \In prof:\ent, \In freq:\ent, \Out b :\bool}{}
		\pre{\True}
		\post{b \Iff esSe\tilde{n}alValida(s,prof,freq)}
	\end{proc}
\end{addmargin}
\vspace{0.5cm}

\begin{addmargin}[4em]{0em}
	\auxpred{esSe\tilde{n}alValida}{\In s :\TLista{\ent}, \In prof:\ent, \In freq:\ent}{\\
		(freq=8 \lor freq=32) \y \\
		(prof=8 \lor prof=16 \lor prof=32) \y \\
		duracion(s, freq) > 1 \y \\
		respetaProfundidad(s, prof) \\
	}
\end{addmargin}
\vspace{0.5cm}

\begin{addmargin}[4em]{0em}
	% Devuelve duracion en segundos
	\aux{duracion}{s :\TLista{\ent}, freq:\ent}{\float}{(\IfThenElse{freq=0}{0}{\frac{|s|}{1000*freq}})}
\end{addmargin}
\vspace{0.5cm}

\begin{addmargin}[4em]{0em}
	\auxpred{respetaProfundidad}{s :\TLista{\ent}, prof :\ent}{\\
		(\forall i :\ent)( 0 \leq i < |s|) \implicaLuego (-2^{prof-1} \leq s[i] \leq 2^{prof-1}-1)\\
	}
\end{addmargin}
\newpage



\addcontentsline{toc}{section}{Ejercicio 2}
\section*{Ejercicio 2}
\vspace{1cm}
\begin{addmargin}[4em]{0em}
	\begin{proc}{seEnojó?}{\In señal :\TLista{\ent}, \In umbral:\ent, \In prof:\ent, \In freq:\ent, \Out b :\bool}{}
		\pre{esSe\tilde{n}alValida(se\tilde{n}al,prof,freq) \y -2^{prof-1} \leq umbral \leq 2^{prof-1}-1)}
		\post{ \\
			b \Iff (\exists subse\tilde{n}al :\TLista{\ent})( \\
				incluida(se\tilde{n}al, subse\tilde{n}al) \y \\
				duracion(subse\tilde{n}al, freq) > 5 \y \\
				superaUmbral(subse\tilde{n}al, umbral)
			) \\
		}
	\end{proc}
\end{addmargin}
\vspace{0.5cm}

\begin{addmargin}[4em]{0em}
	\auxpred{incluida}{s :\TLista{\ent}, subseq :\TLista{\ent}}{\\
		(\exists i :\ent)(0 \leq i \leq |s|-|subseq|) \yLuego \\
		(\forall j :\ent)(0 \leq j < |subseq| \implicaLuego subseq[j]=s[j+i]) \\
	}
\end{addmargin}
\vspace{0.5cm}

\begin{addmargin}[4em]{0em}
	% Devuelve duracion en segundos (freq no es cero por la Pre)
	\aux{duracion}{s :\TLista{\ent}, freq:\ent}{\float}{\frac{|s|}{1000*freq}}
\end{addmargin}
\vspace{0.5cm}

\begin{addmargin}[4em]{0em}
	\auxpred{superaUmbral}{s :\TLista{\ent}, umbral :\ent}{\\
		(\forall i :\ent)(0 \leq i < |s|) \implicaLuego s[i]>umbral) \\
	}
\end{addmargin}
\newpage



\addcontentsline{toc}{section}{Ejercicio 3}
\section*{Ejercicio 3}
\vspace{1cm}
\begin{addmargin}[4em]{0em}
	\begin{proc}{esReunionValida}{\In r :reunion, \In prof:\ent, \In freq:\ent, \Out result :\bool}{}
		\pre{\True}
		\post{result \Iff esReunion(r, prof, freq)}
	\end{proc}
\end{addmargin}
\vspace{0.5cm}

\begin{addmargin}[4em]{0em}
	\auxpred{esReunion}{\In r :reunion, \In prof:\ent, \In freq:\ent}{\\
		|r|>0 \yLuego todasSonSe\tilde{n}ales(r) \y\\
		muestrasDeIgualTama\tilde{n}o(r)  \y \\
		unaSe\tilde{n}alPorPersona(r) \\
		}
\end{addmargin}
\vspace{0.5cm}

\begin{addmargin}[4em]{0em}
	\auxpred{todasSonSe\tilde{n}ales}{\In r :reunion}{
		(\forall i:\ent)(0 \leq i < |r| \implicaLuego esSe\tilde{n}alValida(r[i]_{0})
		}
\end{addmargin}
\vspace{0.5cm}

\begin{addmargin}[4em]{0em}
	\auxpred{muestrasDeIgualTama\tilde{n}o}{\In r :reunion}{\\
		(\forall i :\ent)(0 \leq i < |r|) \implicaLuego |(r[0])_0| = |(r[i])_0| \\
	}
\end{addmargin}
\vspace{0.5cm}

\begin{addmargin}[4em]{0em}
	\auxpred{unaSe\tilde{n}alPorPersona}{\In r :reunion}{\\
		(\forall i :\ent)((\forall j :\ent)((0 \leq i < |r|) \y (0\leq j < |r|) \y (i \neq j) \implicaLuego(\\
		(r[i])_1 \neq (r[j])_1 \y 0\leq (r[i])_1<|r|)))\\
	}
\end{addmargin}
\newpage



\addcontentsline{toc}{section}{Ejercicio 4}
\section*{Ejercicio 4}
\vspace{1cm}
\begin{addmargin}[4em]{0em}
	\begin{proc}{acelerar}{\Inout r :reunion, \In prof:\ent, \In freq:\ent}{}
		\pre{r = r_0}
		\post{result \Iff (tieneLaMitadDeLasMuestras(r,r_0)  \y \\
			           posicionesImpares(r,r_0)) \\
			}
	\end{proc}
\end{addmargin}
\vspace{0.5cm}

\begin{addmargin}[4em]{0em}
	\auxpred{tieneLaMitadDeLasMuestras}{\In r_0 :reunion, \In r :reunion}{\\
		(\forall i :\ent)(\forall j :\ent)(0 \leq i < |r_0| \y 0 \leq j < |r|) \implicaLuego |(r_0[i])_0| = \frac{|(r[j])_0|}{2} \\
	}
\end{addmargin}
\vspace{0.5cm}

\begin{addmargin}[4em]{0em}
	\auxpred{posicionesImpares}{\In r_0 :reunion, \In r :reunion}{\\
		(\forall i :\ent)(\forall j :\ent)(0 \leq i < |r_0| \y 0 \leq j < |r|) \implicaLuego aux((r_0[i])_0,(r[j])_0) \\
	}
\end{addmargin}
\vspace{0.5cm}

\begin{addmargin}[4em]{0em}
	\auxpred{aux}{\In l :\TLista{\ent}, \In m :\TLista{\ent}}{\\
		(\forall i :\ent)(\forall j :\ent)(0 \leq i < |l| \y 0 \leq j < |m| \y \neg esPar(i)) \implicaLuego l[i] = m[j] \\
	}
\end{addmargin}
\vspace{0.5cm}

\begin{addmargin}[4em]{0em}
	\auxpred{esPar}{\In n:\ent}{\\
		n mod 2 = 0 \\
	}
\end{addmargin}
\newpage



\addcontentsline{toc}{section}{Ejercicio 5}
\section*{Ejercicio 5}
\vspace{0.5cm}
\begin{addmargin}[4em]{0em}
	\begin{proc}{relentizar}{\Inout r :reunion, \In prof:\ent, \In freq:\ent}{}
		\pre{r = r_0}
		\post{result \Iff |r|=|r_0| \yLuego (tieneElDobleDeLasMuestras(r,r_0)  \y \\
			           lasMuestrasExtrasSonElPromedioDeLosPuntosVecinos(r,r_0)) \y \\
								 lasMuestrasNoExtrasSonLasOriginales(r,r_0)) \\
								}
	\end{proc}
\end{addmargin}
\vspace{1cm}

\begin{addmargin}[4em]{0em}
	\auxpred{tieneElDobleDeLasMuestras}{\In r_0 :reunion, \In r :reunion}{\\
		(\forall i :\ent)(\forall j :\ent)(0 \leq i < |r_0| \y 0 \leq j < |r|) \implicaLuego |(r_0[i])_0| = 2|(r[j])_0|-1 \\
	}
\end{addmargin}
\vspace{1cm}

\begin{addmargin}[4em]{0em}
	\auxpred{lasMuestrasExtrasSonElPromedioDeLosPuntosVecinos}{\In r_0 :reunion, \In r :reunion}{\\
		(\forall i :\ent)(0 \leq i < |r_0|) \implicaLuego lasMuestrasImparesSonElPromedio((r_0[i])_0,(r[i])_0) \\
	}
\end{addmargin}
\vspace{1cm}

\begin{addmargin}[4em]{0em}
	\auxpred{lasMuestrasImparesSonElPromedio}{\In l :\TLista{\ent}, \In m :\TLista{\ent}}{\\
		(\forall i :\ent)(0 \leq i < |l|-1) \implicaLuego \left \lfloor{\frac{l[i]+l[i+1]}{2}} \right \rfloor = m[2i+1] \\
	}
\end{addmargin}
\vspace{1cm}

\begin{addmargin}[4em]{0em}
	\auxpred{lasMuestrasNoExtrasSonLasOriginales}{\In r_0 :reunion, \In r :reunion}{\\
		(\forall i :\ent)(0 \leq i < |r_0|) \implicaLuego lasMuestrasParesSonLasOriginales((r_0[i])_0,(r[i])_0) \\
	}
\end{addmargin}
\vspace{1cm}

\begin{addmargin}[4em]{0em}
	\auxpred{lasMuestrasParesSonLasOriginales}{\In l :\TLista{\ent}, \In m :\TLista{\ent}}{\\
		(\forall i :\ent)(0 \leq i < |l|) \implicaLuego l[i] = m[2i] \\
}
\end{addmargin}
\vspace{1cm}
\newpage



\addcontentsline{toc}{section}{Ejercicio 6}
\section*{Ejercicio 6}
\vspace{1cm}
\begin{addmargin}[4em]{0em}
	\begin{proc}{tonosDeVozElevados}{\In r : reunión, \In prof:\ent, \In freq:\ent, \Out hablantes :\TLista{hablante}}{}
		\pre{\\
			esReunion(r, prof, frec)\\
			}
		\post{ \\
			(|r| > 0 \implica |hablantes| > 0) \yLuego \\
			(\forall i :\ent)( i \in hablantes \iff (0 \leq i < |r| \yLuego tieneElTonoMasAlto(i, r)) \\
		}
	\end{proc}

\end{addmargin}
\vspace{0.5cm}


\begin{addmargin}[4em]{0em}
	\auxpred{tieneElTonoMasAlto}{i :\ent, r : reunion}{ \\
		(\exists j :\ent) (0 \leq j < |r| \yLuego r[j]_{1}=i \y (\\
		(\forall k :\ent) (0 \leq k < |r| \implicaLuego tono(r[j]_{0}) \geq tono(r[k]_{0}))))\\
	}

\end{addmargin}
\vspace{0.5cm}

\begin{addmargin}[4em]{0em}
	\aux{tono}{s: señal}{\float}{\sum_{i=0}^{|s|-1} \frac{s[i]}{|s|} }

\end{addmargin}
\newpage




\addcontentsline{toc}{section}{Ejercicio 7}
\section*{Ejercicio 7}
\vspace{1cm}
\begin{addmargin}[4em]{0em}
	\begin{proc}{ordenar}{\Inout r:reunion, \In prof:\ent, \In freq:\ent}{}
		\pre{r=r_{0} \y esReunion(r_{0})}
		\post{estanOrdenados(r) \y sonLaMismaReunion(r_{0},r)}
	\end{proc}
\end{addmargin}
\vspace{0.5cm}

\begin{addmargin}[4em]{0em}
    \auxpred{estanOrdenados}{r:reunion}{\\
    (\forall i:\ent)(0\leq i<|r|-1)\implicaLuego tono(r[i]{0})>tono(r[i+1]{0})\\
    }
\end{addmargin}
\vspace{0.5cm}

\begin{addmargin}[4em]{0em}
    \auxpred{sonLaMismaReunion}{r_{0}:reunion, r: reunion}{\\
	(\forall s: se\tilde{n}al)(s \in r_{0} \iff s \in r)\\
    }
\end{addmargin}
\newpage



\addcontentsline{toc}{section}{Ejercicio 8}
\section*{Ejercicio 8}
\vspace{0.5cm}
\begin{addmargin}[4em]{0em}
	\begin{proc}{silencios}{\In s : señal, \In prof:\ent, \In freq:\ent, \In umbral:\ent, \Out intervalos :\TLista{intervalo}}{}
		\pre{\\
			duracion(se\tilde(n)al, freq) > 0 \y\\
			respetaProfundidad(se\tilde{n}al, prof) \y\\
			-2^{prof-1} \leq umbral \leq 2^{prof-1}-1\\
		}
		\post{ \\
			(\forall interv:intervalo)((interv \in intervalos) \Iff (duracionDeIntervalo(interv, freq) > 0.1 \yLuego\\
			tieneLimitesValidos(interv, s) \yLuego\\
			esIntervaloSilencioso(interv, se\tilde{n}, ubmral)))\y \\
			noHayRepetidos(intervalos) \\
		}
	\end{proc}
\end{addmargin}
\vspace{0.5cm}

\begin{addmargin}[4em]{0em}
	% Devuelve duracion en segundos
	\aux{duracion}{s :\TLista{\ent}, freq:\ent}{\float}{(\IfThenElse{freq=0}{0}{\frac{|s|}{1000*freq}})}
\end{addmargin}
\vspace{1cm}

\begin{addmargin}[4em]{0em}
	\auxpred{respetaProfundidad}{s :\TLista{\ent}, prof :\ent}{\\
		(\forall i :\ent)(0 \leq i < |s| \implicaLuego -2^{prof-1} \leq s[i] \leq 2^{prof-1}-1) \\
	}
\end{addmargin}
\vspace{1cm}

\begin{addmargin}[4em]{0em}
	% Devuelve duracion en segundos
	\aux{duracionDeIntervalo}{interv :intervalo, freq:\ent}{\float}{(\IfThenElse{freq=0}{0}{\frac{interv_1-interv_0}{1000*freq}})}
\end{addmargin}
\vspace{1cm}

\begin{addmargin}[4em]{0em}
	\auxpred{tieneLimitesValidos}{interv :intervalo, s : se\tilde{n}al}{\\
		0 \leq interv_0 \leq interv_1 < |s| \\
	}
\end{addmargin}
\vspace{1cm}

\begin{addmargin}[4em]{0em}
	\auxpred{esIntervaloSilencioso}{interv :intervalo,  s : se\tilde{n}al, umbral:\ent}{\\
		(\forall i :\ent)(interv_1 \leq i \leq interv_2 \implicaLuego (|s[i]| \leq umbral)) \y \\
		(interv_0=0 \oLuego |s[interv_0-1]| > umbral) \y (interv_1=|s|-1 \oLuego |s[interv_1+1]| > umbral)\\
	}
\end{addmargin}
\vspace{1cm}

\begin{addmargin}[4em]{0em}
	\auxpred{noHayRepetidos}{interv :\TLista{intervalo}}{\\
		(\forall i,j :\ent)((i\neq j \y 0\leq i,j <|interv|) \implicaLuego (interv[i]_0 \neq interv[j]_0 \lor interv[i]_1 \neq interv[j]_1)) \\
	}
\end{addmargin}
\vspace{1cm}
\newpage



\addcontentsline{toc}{section}{Ejercicio 9}
\section*{Ejercicio 9}
\vspace{0.5cm}
\begin{addmargin}[4em]{0em}
	\begin{proc}{hablantesSuperpuestos}{\In r :reunion, \In prof:\ent, \In freq:\ent, \In umbral:\ent, \Out result :\bool}{}
		\pre{\\
			-2^{prof-1} \leq umbral \leq 2^{prof-1}-1\\
		}
		\post{ \\
			result \Iff ((\exists interv:intervalo)(duracionDeIntervalo(interv, freq) > 0.1 \yLuego\\
			((\exists s_1 : se\tilde{n}al)(\exists s_2 : se\tilde{n}al)(se\tilde{n}alEnReunion(s_1, r) \y (se\tilde{n}alEnReunion(s_2, r)) \yLuego \\
			(tieneLimitesValidos(interv, s_1) \y tieneLimitesValidos(interv, s_2)) \yLuego\\
			(esIntervaloRuidoso(interv, s_1, umbral) \y  esIntervaloRuidoso(interv, s_2, umbral))))) \\
		}
	\end{proc}
\end{addmargin}
\vspace{0.5cm}

\begin{addmargin}[4em]{0em}
	% Devuelve duracion en segundos
	\aux{duracionDeIntervalo}{\In interv :intervalo,\In freq:\ent}{\float}{(\IfThenElse{freq=0}{0}{\frac{interv_1-interv_0}{1000*freq}})}
\end{addmargin}
\vspace{1cm}

\begin{addmargin}[4em]{0em}
	\auxpred{se\tilde{n}alEnReunion}{\In s : se\tilde{n}al,\In r : reunion}{\\
		(\exists i :\ent)(0 \leq i < |r| \y |s| = |(reunion[i])0|) \yLuego (\forall j :\ent)(0 \leq j < |s| \implicaLuego s[j] = (reunion[i])_0[j]) \\
	}
\end{addmargin}
\vspace{1cm}

\begin{addmargin}[4em]{0em}
	\auxpred{tieneLimitesValidos}{\In interv :intervalo,\In s : se\tilde{n}al}{\\
		0 \leq interv_0 \leq interv_1 < |s| \\
	}
\end{addmargin}
\vspace{1cm}

\begin{addmargin}[4em]{0em}
	\auxpred{esIntervaloRuidoso}{\In interv :intervalo,\In  s : se\tilde{n}al,\In umbral:\ent}{\\
		(\forall i :\ent)(interv_1 \leq i \leq interv_2 \implicaLuego (|s[i]| > umbral)) \y \\
		(interv_0=0 \oLuego |s[interv_0-1]| \leq umbral) \y (interv_1=|s|-1 \oLuego |s[interv_1+1]| \leq umbral)\\
	}
\end{addmargin}
\vspace{1cm}
\newpage



\addcontentsline{toc}{section}{Ejercicio 10}
\section*{Ejercicio 10}
\vspace{1cm}
\begin{addmargin}[4em]{0em}
	\begin{proc}{reconstruir}{\In s: señal, \In frec: \ent, \In prof:\ent, \Out result: señal}{}
		\pre{esSe\tilde{n}al(s, prof, frec) \y tieneAlMenosDosValoresNoNulos(s)}
		\post{ \\
			|s|=|result| \yLuego (\\
			(\forall i : \ent)(0\leq i<|s|)\implicaLuego (\\
			(s[i] = result[i] \y s[i]\neq 0) \vee seReparo(i,s,result)))\\
		}
	\end{proc}
\end{addmargin}
\vspace{0.5cm}

\begin{addmargin}[4em]{0em}
	\auxpred{seReparo}{\In i: \ent, \In s :se\tilde{n}al, \In result :se\tilde{n}al}{\\
		(\exists a:\ent)(\exists b:\ent)(0 \leq a < b < |s| \yLuego sonLosMasCercanos(a,b,i,s) \yLuego\\
		result[i] = \left \lfloor{\frac{a + b}{2}} \right \rfloor)\\
	}
\end{addmargin}
\vspace{0.5cm}

\begin{addmargin}[4em]{0em}
	\auxpred{sonLosMasCercanos}{\In a: \ent, \In b :\ent, \In i :\ent, \In s: se\tilde{n}al}{\\
		(\forall k : \ent)((0 \leq k < |s| \yLuego (|k-i|<|a-i| \o |k-i|<|b-i|)) \implica s[k]=0)\\
	}
\end{addmargin}
\vspace{0.5cm}

\begin{addmargin}[4em]{0em}
	\auxpred{tieneAlMenosDosValoresNoNulos}{\In s: se\tilde{n}al}{\\
		(\exists a : \ent)(\exists b : \ent)((0 \leq a < |s| \y 0 \leq b < |s| \y a\neq b) \yLuego\\
		s[a]\neq 0 \y s[b]\neq 0)\\
	}
\end{addmargin}


\end{document}





				