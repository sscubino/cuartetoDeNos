\documentclass{article}
\input{Algo1Macros}
\usepackage{caratula}
\usepackage{scrextend}


\begin{document}

%Carátula
\titulo{TP 1 - Reuniones Remotas}
\subtitulo{Grupo 3}
\fecha{22 de Mayo de 2020}
\materia{Algoritmos y Estructuras de Datos 1}
\integrante{González Narvarte, Francisco}{519/15}{francisco13\_95@live.com}
\integrante{Giménez, Iván Manuel}{374/18}{ivangimenez8727@gmail.com}
\integrante{Demare, Matías Nicolás}{762/19}{matiasdemare@gmail.com}
\integrante{Cubino, Santiago}{829/19}{sscubino@gmail.com}
\maketitle

%Creación de índice
\tableofcontents
\newpage

% End carátula


{\Large Ejercicio 1\par}
\vspace{0.5cm}

\begin{addmargin}[4em]{0em}
	\begin{proc}{esSeñal}{\In s :\TLista{\ent}, \In prof:\ent, \In freq:\ent, \Out b :\bool}{}
		\pre{\True}
		\post{b \Iff (freq=8 \lor freq=32) \y \\
				(prof=8 \lor prof=16 \lor prof=32) \y \\
				duracion(s, freq) > 1 \y \\
				respetaProfundidad(s, prof) \\
				}
	\end{proc}
\end{addmargin}
\vspace{0.5cm}

\begin{addmargin}[4em]{0em}
	% Devuelve duracion en segundos
	\aux{duracion}{s :\TLista{\ent}, freq:\ent}{\float}{(\IfThenElse{freq=0}{0}{\frac{|s|}{1000*freq}})}
\end{addmargin}
\vspace{1cm}

\begin{addmargin}[4em]{0em}
	\auxpred{respetaProfundidad}{s :\TLista{\ent}, prof :\ent}{\\
		(\forall i :\ent)(0 \leq i < |s| \implicaLuego -2^{prof-1} \leq s[i] \leq 2^{prof-1}-1) \\
	}
\end{addmargin}



%\newpage
\vspace{1cm}

{\Large Ejercicio 2\par}
\vspace{0.5cm}

\begin{addmargin}[4em]{0em}
	\begin{proc}{seEnojó?}{\In señal :\TLista{\ent}, \In umbral:\ent, \In prof:\ent, \In freq:\ent, \Out b :\bool}{}
		\pre{-2^{prof-1} \leq umbral \leq 2^{prof-1}-1 \y freq > 0}
		\post{ \\
			b \Iff (\exists subse\tilde{n}al :\TLista{\ent})( \\
				incluida(se\tilde{n}al, subse\tilde{n}al) \y \\
				duracion(subse\tilde{n}al, freq) > 5 \y \\
				superaUmbral(subse\tilde{n}al, umbral)
			) \\
		}
	\end{proc}
\end{addmargin}
\vspace{1cm}

\begin{addmargin}[4em]{0em}
	\auxpred{incluida}{s :\TLista{\ent}, subseq :\TLista{\ent}}{\\
		(\exists i :\ent)(0 \leq i \leq |s|-|subseq| \yLuego \\
		(\forall j :\ent)(0 \leq j < |subseq| \implicaLuego subseq[j]=s[j+i])) \\
	}
\end{addmargin}
\vspace{1cm}

\begin{addmargin}[4em]{0em}
	% Devuelve duracion en segundos
	\aux{duracion}{s :\TLista{\ent}, freq:\ent}{\float}{\frac{|s|}{1000*freq}}
\end{addmargin}
\vspace{1cm}

\begin{addmargin}[4em]{0em}
	\auxpred{superaUmbral}{s :\TLista{\ent}, umbral :\ent}{\\
		(\forall i :\ent)(0 \leq i < |s| \implicaLuego s[i]>umbral) \\
	}
\end{addmargin}

\newpage
{\Large Ejercicio 3\par}

\newpage
{\Large Ejercicio 6\par}
\vspace{0.5cm}

\begin{addmargin}[4em]{0em}
	\begin{proc}{tonosDeVozElevados}{\In r : reunión, \In prof:\ent, \In freq:\ent, \Out hablantes :\TLista{hablante}}{}
		\pre{\True}\\
		\post{ \\
			(|r| > 0 \implica |hablantes| > 0) \yLuego \\
			(\forall i :\ent)( i \in hablantes \implicaLuego 0 \leq i < |r| \yLuego tieneElTonoMasAlto(i, r)) \\
		}
	\end{proc}

\end{addmargin}

\vspace{0.5cm}

\begin{addmargin}[4em]{0em}
	\auxpred{tieneElTonoMasAlto}{i :\ent, r : reunion}{ \\
		(\exists j :\ent)(0 \leq j < |r| \yLuego r[j]_{1}=i \y (\\
		(\forall k :\ent) (0 \leq k < |r| \implicaLuego tono(r[j]_{0}) \geq tono(r[k]_{0})))\\
	}

\end{addmargin}

\vspace{0.5cm}

\begin{addmargin}[4em]{0em}
	\aux{tono}{s: se\tilde{n}al}{\float}{\sum_{i=0}^{|s|-1}\frac{s[i]}{|s|} $}

\end{addmargin}


\end{document}