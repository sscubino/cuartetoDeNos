\documentclass{article}
\input{Algo1Macros}
\usepackage{caratula}
\usepackage{scrextend}


\begin{document}

%Carátula
\titulo{TP 1 - Reuniones Remotas}
\subtitulo{Grupo 3}
\fecha{22 de Mayo de 2020}
\materia{Algoritmos y Estructuras de Datos 1}
\integrante{González Narvarte, Francisco}{519/15}{francisco13\_95@live.com}
\integrante{Giménez, Iván Manuel}{374/18}{ivangimenez8727@gmail.com}
\integrante{Demare, Matías Nicolás}{762/19}{matiasdemare@gmail.com}
\integrante{Cubino, Santiago}{829/19}{sscubino@gmail.com}
\maketitle

%Creación de índice
\tableofcontents
\newpage

% End carátula


{\Large Ejercicio 1\par}
\vspace{0.5cm}

\begin{addmargin}[4em]{0em}
	\begin{proc}{esSeñal}{\In s :\TLista{\ent}, \In prof:\ent, \In freq:\ent, \Out b :\bool}{}
		\pre{\True}
		\post{b \Iff esSe\tilde{n}alValida(s,prof,freq)}
	\end{proc}
\end{addmargin}
\vspace{0.5cm}

\begin{addmargin}[4em]{0em}
	\auxpred{esSe\tilde{n}alValida}{\In s :\TLista{\ent}, \In prof:\ent, \In freq:\ent}{\\
		(freq=8 \lor freq=32) \y \\
		(prof=8 \lor prof=16 \lor prof=32) \y \\
		duracion(s, freq) > 1 \y \\
		respetaProfundidad(s, prof) \\
	}
\end{addmargin}
\vspace{0.5cm}

\begin{addmargin}[4em]{0em}
	% Devuelve duracion en segundos
	\aux{duracion}{s :\TLista{\ent}, freq:\ent}{\float}{(\IfThenElse{freq=0}{0}{\frac{|s|}{1000*freq}})}
\end{addmargin}
\vspace{1cm}

\begin{addmargin}[4em]{0em}
	\auxpred{respetaProfundidad}{s :\TLista{\ent}, prof :\ent}{\\
		(\forall i :\ent)( 0 \leq i < |s|) \implicaLuego (-2^{prof-1} \leq s[i] \leq 2^{prof-1}-1)\\
	}
\end{addmargin}
\vspace{1cm}

\begin{addmargin}[4em]{0em}
	\auxpred{entre}{\In a :\ent,\In i :\ent, \In b:\ent}{a \leq i \y i < b}
\end{addmargin}
\vspace{1cm}







\newpage

{\Large Ejercicio 2\par}
\vspace{0.5cm}

\begin{addmargin}[4em]{0em}
	\begin{proc}{seEnojó?}{\In señal :\TLista{\ent}, \In umbral:\ent, \In prof:\ent, \In freq:\ent, \Out b :\bool}{}
		\pre{esSe\tilde{n}alValida(se\tilde{n}al,prof,freq) \y -2^{prof-1} \leq umbral \leq 2^{prof-1}-1)}
		\post{ \\
			b \Iff (\exists subse\tilde{n}al :\TLista{\ent})( \\
				incluida(se\tilde{n}al, subse\tilde{n}al) \y \\
				duracion(subse\tilde{n}al, freq) > 5 \y \\
				superaUmbral(subse\tilde{n}al, umbral)
			) \\
		}
	\end{proc}
\end{addmargin}
\vspace{1cm}

\begin{addmargin}[4em]{0em}
	\auxpred{incluida}{s :\TLista{\ent}, subseq :\TLista{\ent}}{\\
		(\exists i :\ent)(0 \leq i \leq |s|-|subseq|) \yLuego \\
		(\forall j :\ent)(0 \leq j < |subseq| \implicaLuego subseq[j]=s[j+i]) \\
	}
\end{addmargin}
\vspace{1cm}

\begin{addmargin}[4em]{0em}
	% Devuelve duracion en segundos (freq no es cero por la Pre)
	\aux{duracion}{s :\TLista{\ent}, freq:\ent}{\float}{\frac{|s|}{1000*freq}}
\end{addmargin}
\vspace{1cm}

\begin{addmargin}[4em]{0em}
	\auxpred{superaUmbral}{s :\TLista{\ent}, umbral :\ent}{\\
		(\forall i :\ent)(0 \leq i < |s|) \implicaLuego s[i]>umbral) \\
	}
\end{addmargin}








\newpage
{\Large Ejercicio 3\par}
\vspace{0.5cm}

\begin{addmargin}[4em]{0em}
	\begin{proc}{esReunionValida}{\In r :reunion, \In prof:\ent, \In freq:\ent, \Out result :\bool}{}
		\pre{\True}
		\post{result \Iff esReunion(r, prof, freq)}
	\end{proc}
\end{addmargin}
\vspace{1cm}

\begin{addmargin}[4em]{0em}
	\auxpred{esReunion}{\In r :reunion, \In prof:\ent, \In freq:\ent}{\\
		|r|>0 \yLuego todasSonSe\tilde{n}ales(r) \y\\
		muestrasDeIgualTama\tilde{n}o(r)  \y \\
		unaSe\tilde{n}alPorPersona(r) \\
		}
\end{addmargin}
\vspace{1cm}

\begin{addmargin}[4em]{0em}
	\auxpred{todasSonSe\tilde{n}ales}{\In r :reunion}{
		(\forall i:\ent)(0 \leq i < |r| \implicaLuego esSe\tilde{n}alValida(r[i]_{0})
		}
\end{addmargin}
\vspace{1cm}

\begin{addmargin}[4em]{0em}
	\auxpred{muestrasDeIgualTama\tilde{n}o}{\In r :reunion}{\\
		(\forall i :\ent)(0 \leq i < |r|) \implicaLuego |(r[0])_0| = |(r[i])_0| \\
	}
\end{addmargin}
\vspace{1cm}

\begin{addmargin}[4em]{0em}
	\auxpred{unaSe\tilde{n}alPorPersona}{\In r :reunion}{\\
		(\forall i :\ent)(\forall j :\ent)((0 \leq i < |r|) \y (0\leq j < |r|) \y (i \neq j) \implicaLuego(\\
		(r[i])_1 \neq (r[j])_1 \y entre(0,(r[i])_1 , |r|))\\
	}
\end{addmargin}
\vspace{1cm}







\newpage
{\Large Ejercicio 4\par}
\vspace{0.5cm}

\begin{addmargin}[4em]{0em}
	\begin{proc}{acelerar}{\Inout r :reunion, \In prof:\ent, \In freq:\ent}{}
		\pre{r = r_0 \y esReunionValida(r)}
		\post{result \Iff |r|=|r_0| \yLuego (tieneLaMitadDeLasMuestras(r,r_0)  \y \\
			           posicionesImpares(r,r_0)) \\
								}
	\end{proc}
\end{addmargin}
\vspace{1cm}

\begin{addmargin}[4em]{0em}
	\auxpred{tieneLaMitadDeLasMuestras}{\In r_0 :reunion, \In r :reunion}{\\
		(\forall i :\ent)(\forall j :\ent)(0 \leq i < |r_0| \y 0 \leq j < |r|) \implicaLuego 
		\left \lfloor{\frac{|(r_0[j])_0|}{2}} \right \rfloor = |(r[i])_0| \\
	}
\end{addmargin}
\vspace{1cm}

\begin{addmargin}[4em]{0em}
	\auxpred{posicionesImpares}{\In r_0 :reunion, \In r :reunion}{\\
		(\forall i :\ent)(0 \leq i < |r_0|) \implicaLuego tomaLasPosicionesImpares((r_0[i])_0,(r[i])_0) \\
	}
\end{addmargin}
\vspace{1cm}

\begin{addmargin}[4em]{0em}
	\auxpred{tomaLasPosicionesImpares}{\In l :\TLista{\ent}, \In m :\TLista{\ent}}{\\
		(\forall i :\ent)(1 \leq i \leq |m|) \implicaLuego l[2i-1] = m[i-1] \\
	}
\end{addmargin}
\vspace{1cm}




\newpage
{\Large Ejercicio 5\par}
\vspace{0.5cm}

\begin{addmargin}[4em]{0em}
	\begin{proc}{relentizar}{\Inout r :reunion, \In prof:\ent, \In freq:\ent}{}
		\pre{r = r_0}
		\post{result \Iff |r|=|r_0| \yLuego (tieneElDobleDeLasMuestras(r,r_0)  \y \\
			           lasMuestrasExtrasSonElPromedioDeLosPuntosVecinos(r,r_0)) \y \\
								 lasMuestrasNoExtrasSonLasOriginales(r,r_0)) \\
								}
	\end{proc}
\end{addmargin}
\vspace{1cm}

\begin{addmargin}[4em]{0em}
	\auxpred{tieneElDobleDeLasMuestras}{\In r_0 :reunion, \In r :reunion}{\\
		(\forall i :\ent)(\forall j :\ent)(0 \leq i < |r_0| \y 0 \leq j < |r|) \implicaLuego |(r_0[i])_0| = 2|(r[j])_0|-1 \\
	}
\end{addmargin}
\vspace{1cm}

\begin{addmargin}[4em]{0em}
	\auxpred{lasMuestrasExtrasSonElPromedioDeLosPuntosVecinos}{\In r_0 :reunion, \In r :reunion}{\\
		(\forall i :\ent)(0 \leq i < |r_0|) \implicaLuego lasMuestrasImparesSonElPromedio((r_0[i])_0,(r[i])_0) \\
	}
\end{addmargin}
\vspace{1cm}

\begin{addmargin}[4em]{0em}
	\auxpred{lasMuestrasImparesSonElPromedio}{\In l :\TLista{\ent}, \In m :\TLista{\ent}}{\\
		(\forall i :\ent)(0 \leq i < |l|-1) \implicaLuego \left \lfloor{\frac{l[i]+l[i+1]}{2}} \right \rfloor = m[2i+1] \\
	}
\end{addmargin}
\vspace{1cm}

\begin{addmargin}[4em]{0em}
	\auxpred{lasMuestrasNoExtrasSonLasOriginales}{\In r_0 :reunion, \In r :reunion}{\\
		(\forall i :\ent)(0 \leq i < |r_0|) \implicaLuego lasMuestrasParesSonLasOriginales((r_0[i])_0,(r[i])_0) \\
	}
\end{addmargin}
\vspace{1cm}

\begin{addmargin}[4em]{0em}
	\auxpred{lasMuestrasParesSonLasOriginales}{\In l :\TLista{\ent}, \In m :\TLista{\ent}}{\\
		(\forall i :\ent)(0 \leq i < |l|) \implicaLuego l[i] = m[2i] \\
}
\end{addmargin}
\vspace{1cm}



\newpage
{\Large Ejercicio 6\par}
\vspace{0.5cm}

\begin{addmargin}[4em]{0em}
	\begin{proc}{tonosDeVozElevados}{\In r : reunión, \In prof:\ent, \In freq:\ent, \Out hablantes :\TLista{hablante}}{}
		\pre{\\
			(\forall i:\ent)(0\leq i < |r| \implicaLuego esSe\tilde{n}al(r[i]_{0}, prof, freq) \y (\\
			(\forall i:\ent)(\forall j:\ent)(0\leq i,j < |r| \implicaLuego 0\leq s[i]_{1},s[j]_{1}<|s| \y s[i]_{1}=s[j]_{1} \leftrightarrow i=j))\\
			%verifica que los hablantes esten numerados de 0 a |s|-1, sin repetir
			}
		\post{ \\
			(|r| > 0 \implica |hablantes| > 0) \yLuego \\
			(\forall i :\ent)( i \in hablantes \implicaLuego 0 \leq i < |r| \yLuego tieneElTonoMasAlto(i, r)) \\
		}
	\end{proc}

\end{addmargin}

\vspace{0.5cm}


\begin{addmargin}[4em]{0em}
	\auxpred{tieneElTonoMasAlto}{i :\ent, r : reunion}{ \\
		(\exists j :\ent)(0 \leq j < |r| \yLuego r[j]_{1}=i \y (\\
		(\forall k :\ent) (0 \leq k < |r| \implicaLuego tono(r[j]_{0}) \geq tono(r[k]_{0})))\\
	}

\end{addmargin}

\vspace{0.5cm}

\begin{addmargin}[4em]{0em}
	\aux{tono}{s: señal}{\float}{\sum_{i=0}^{|s|-1} \frac{s[i]}{|s|} }

\end{addmargin}







\newpage



{\Large Ejercicio 7}
\vspace{0.5cm}
\begin{addmargin}[4cm]{0em}
	\begin{proc}{ordenar}{\Inout r:reunion, \In prof:\ent, \In freq:\ent}{}
		\pre{freq > 0 \yLuego respetaProfundidad2(r) \y duranLoMismo(r,freq)}
		\post{estanOrdenados(r)}
	\end{proc}
\end{addmargin}

\vspace{1cm}

\begin{addmargin}[4em]{0em}
    \auxpred{estanOrdenados}{r:reunion}{\
    (\forall i:\ent)(0<i<|r|)\implicaLuego promedio(r[i]{0})>promedio(r[i-1]{0})\
    }
\end{addmargin}
\vspace{0.5cm}
\begin{addmargin}[4em]{0em}
    \aux{promedio}{s:\TLista{\ent}}{\float}{\sum{i=0}^{|s|-1}\frac{s[i]}{|s|}}
\end{addmargin}
\vspace{0.5cm}
\begin{addmargin}[4em]{0em}
	\auxpred{duranLoMismo}{r:reunion, freq:\ent}{\\
		(\forall i:\ent)(0<i<|r|)\implicaLuego \frac{|r[i]{0}|}{freq1000}=\frac{|r[i-1]_{0}|}{freq1000}\\
	}
\end{addmargin}
\vspace{0.5cm}
\begin{addmargin}[4em]{0em}
    \auxpred{respetaProfundidad}{s:\TLista{\ent}, prof:\ent}{\\
		(\forall i:\ent)(0<i<|s|)\implicaLuego -2^{prof-1} \leq s[i] \leq 2^{prof-1}-1)\\
	}
\end{addmargin}
\vspace{0.5cm}
\begin{addmargin}[4em]{0em}
    \auxpred {respetaProfundidad2}{r:reunion}{\\
		(\forall i:\ent)(0<=i<|r|)\implicaLuego respetaProfundidad(r[i]_{0})\\
	}
\end{addmargin}


\newpage
{\Large Ejercicio 9\par}
\vspace{0.5cm}

\begin{addmargin}[4em]{0em}
	\begin{proc}{hablantesSuperpuestos}{\In r :reunion, \In prof:\ent, \In freq:\ent, \In umbral:\ent, \Out result :\bool}{}
		\pre{\\
			-2^{prof-1} \leq umbral \leq 2^{prof-1}-1\\
		}
		\post{ \\
			result \Iff ((\exists interv:intervalo)(duracionDeIntervalo(interv, freq) > 0.1 \yLuego\\
			((\exists s_1 : se\tilde{n}al)(\exists s_2 : se\tilde{n}al)(se\tilde{n}alEnReunion(s_1, r) \y (se\tilde{n}alEnReunion(s_2, r)) \yLuego \\
			(tieneLimitesValidos(interv, s_1) \y tieneLimitesValidos(interv, s_2)) \yLuego\\
			(esIntervaloRuidoso(interv, s_1, umbral) \y  esIntervaloRuidoso(interv, s_2, umbral))))) \\
		}
	\end{proc}
\end{addmargin}
\vspace{0.5cm}

\begin{addmargin}[4em]{0em}
	% Devuelve duracion en segundos
	\aux{duracionDeIntervalo}{\In interv :intervalo,\In freq:\ent}{\float}{(\IfThenElse{freq=0}{0}{\frac{interv_1-interv_0}{1000*freq}})}
\end{addmargin}
\vspace{1cm}

\begin{addmargin}[4em]{0em}
	\auxpred{se\tilde{n}alEnReunion}{\In s : se\tilde{n}al,\In r : reunion}{\\
		(\exists i :\ent)(0 \leq i < |r| \y |s| = |(reunion[i])0|) \yLuego (\forall j :\ent)(0 \leq j < |s| \implicaLuego s[j] = (reunion[i])_0[j]) \\
	}
\end{addmargin}
\vspace{1cm}

\begin{addmargin}[4em]{0em}
	\auxpred{tieneLimitesValidos}{\In interv :intervalo,\In s : se\tilde{n}al}{\\
		0 \leq interv_0 \leq interv_1 < |s| \\
	}
\end{addmargin}
\vspace{1cm}

\begin{addmargin}[4em]{0em}
	\auxpred{esIntervaloRuidoso}{\In interv :intervalo,\In  s : se\tilde{n}al,\In umbral:\ent}{\\
		(\forall i :\ent)(interv_1 \leq i \leq interv_2 \implicaLuego (|s[i]| > umbral)) \y \\
		(interv_0=0 \oLuego |s[interv_0-1]| \leq umbral) \y (interv_1=|s|-1 \oLuego |s[interv_1+1]| \leq umbral)\\
	}
\end{addmargin}
\vspace{1cm}



\newpage

{\Large Ejercicio 10\par}
\vspace{1cm}

\begin{addmargin}[4em]{0em}
	\begin{proc}{reconstruir}{\In s: señal, \In frec: \ent, \In prof:\ent, \Out result: señal}{}
		\pre{esSe\tilde{n}al(s, prof, frec)}
		\post{ \\
			|s|=|result| \yLuego (\\
			(\forall i : \ent)(0\leq i<|s|)\implicaLuego\\
			s[i] = result[i] \vee seReparo(i,s,result))\\
		}
	\end{proc}
\end{addmargin}

\vspace{0.5cm}

\begin{addmargin}[4em]{0em}
	\auxpred{seReparo}{\In i: \ent, \In s :se\tilde{n}al, \In result :se\tilde{n}al}{\\
		(\exists a,b:\ent)(0 \leq a < i < b < |s| \yLuego esIntervaloDeCeros(a,b,s) \yLuego\\
		result[i] = \left \lfloor{\frac{a + b}{2}} \right \rfloor)\\
	}
\end{addmargin}

\vspace{0.5cm}

\begin{addmargin}[4em]{0em}
	\auxpred{esIntervaloDeCeros}{\In a: \ent, \In b :\ent, \In s: se\tilde{n}al}{\\
		(\forall i : \ent)(a < i < b \implicaLuego s[i]=0)\\
	}
\end{addmargin}

\end{document}











\begin{addmargin}[4em]{0em}
	\auxpred{pertenece}{\In elem :T,\In l:\TLista{T}}{\\
		(\exists i:\ent)(entre(0,i,|l|)\yLuego l[i]=elem)\\
	}
\end{addmargin}




				