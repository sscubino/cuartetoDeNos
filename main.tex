\documentclass{article}
\input{Algo1Macros}
\usepackage{caratula}
\usepackage{scrextend}


\begin{document}

%Carátula
\titulo{TP 1 - Reuniones Remotas}
\subtitulo{Grupo 3}
\fecha{22 de Mayo de 2020}
\materia{Algoritmos y Estructuras de Datos 1}
\integrante{González Narvarte, Francisco}{519/15}{francisco13\_95@live.com}
\integrante{Giménez, Iván Manuel}{374/18}{ivangimenez8727@gmail.com}
\integrante{Demare, Matías Nicolás}{762/19}{matiasdemare@gmail.com}
\integrante{Cubino, Santiago}{829/19}{sscubino@gmail.com}
\maketitle

%Creación de índice
\tableofcontents
\newpage

% End carátula


{\Large Ejercicio 1\par}
\vspace{0.5cm}

\begin{addmargin}[4em]{0em}
	\begin{proc}{esSeñal}{\In s :\TLista{\ent}, \In prof:\ent, \In freq:\ent, \Out b :\bool}{}
		\pre{\True}
		\post{b \Iff (freq=8 \lor freq=32) \y \\
				(prof=8 \lor prof=16 \lor prof=32) \y \\
				duracion(s, freq) > 1 \y \\
				respetaProfundidad(s, prof) \\
				}
	\end{proc}
\end{addmargin}
\vspace{1cm}

\begin{addmargin}[4em]{0em}
	% Devuelve duracion en segundos
	\aux{duracion}{s :\TLista{\ent}, freq:\ent}{\float}{(\IfThenElse{freq=0}{0}{\frac{|s|}{1000*freq}})}
\end{addmargin}
\vspace{1cm}

\begin{addmargin}[4em]{0em}
	\auxpred{respetaProfundidad}{s :\TLista{\ent}, prof :\ent}{\\
		(\forall i :\ent)(0 \leq i < |s| \implicaLuego -2^{prof-1} \leq s[i] \leq 2^{prof-1}-1) \\
	}
\end{addmargin}



\newpage

{\Large Ejercicio 2\par}
\vspace{0.5cm}

\begin{addmargin}[4em]{0em}
	\begin{proc}{seEnojó?}{\In señal :\TLista{\ent}, \In umbral:\ent, \In prof:\ent, \In freq:\ent, \Out b :\bool}{}
		\pre{-2^{prof-1} \leq umbral \leq 2^{prof-1}-1 \y freq > 0}
		\post{ \\
			b \Iff (\exists subse\tilde{n}al :\TLista{\ent})( \\
				incluida(se\tilde{n}al, subse\tilde{n}al) \y \\
				duracion(subse\tilde{n}al, freq) > 5 \y \\
				superaUmbral(subse\tilde{n}al, umbral)
			) \\
		}
	\end{proc}
\end{addmargin}
\vspace{1cm}

\begin{addmargin}[4em]{0em}
	\auxpred{incluida}{s :\TLista{\ent}, subseq :\TLista{\ent}}{\\
		(\exists i :\ent)(0 \leq i \leq |s|-|subseq| \yLuego \\
		(\forall j :\ent)(0 \leq j < |subseq| \implicaLuego subseq[j]=s[j+i])) \\
	}
\end{addmargin}
\vspace{1cm}

\begin{addmargin}[4em]{0em}
	% Devuelve duracion en segundos
	\aux{duracion}{s :\TLista{\ent}, freq:\ent}{\float}{\frac{|s|}{1000*freq}}
\end{addmargin}
\vspace{1cm}

\begin{addmargin}[4em]{0em}
	\auxpred{superaUmbral}{s :\TLista{\ent}, umbral :\ent}{\\
		(\forall i :\ent)(0 \leq i < |s| \implicaLuego s[i]>umbral) \\
	}
\end{addmargin}

\newpage
{\Large Ejercicio 3\par}
\vspace{0.5cm}

\begin{addmargin}[4em]{0em}
	\begin{proc}{esReunionValida}{\In r :reunion, \In prof:\ent, \In freq:\ent, \Out result :\bool}{}
		\pre{\True}
		\post{result \Iff (muestrasDeIgualTama\tilde{n}o(r)  \y \\
			           unaSe\tilde{n}alPorPersona(r)) \\
								}
	\end{proc}
\end{addmargin}
\vspace{1cm}

\begin{addmargin}[4em]{0em}
	\auxpred{muestrasDeIgualTama\tilde{n}o}{\In r :reunion}{\\
		(\forall i :\ent)(0 \leq i < |r|) \implicaLuego |(r[0])_0| = |(r[i])_0| \\
	}
\end{addmargin}
\vspace{1cm}

\begin{addmargin}[4em]{0em}
	\auxpred{unaSe\tilde{n}alPorPersona}{\In r :reunion}{\\
	(\forall i :\ent)(\forall j :\ent)(0 \leq i < |r| \y 0 \leq j < |r|) \implicaLuego (r[i])_1 \neq (r[j])_1 \\
}
\end{addmargin}

\newpage
{\Large Ejercicio 4\par}
\vspace{0.5cm}

\begin{addmargin}[4em]{0em}
	\begin{proc}{acelerar}{\Inout r :reunion, \In prof:\ent, \In freq:\ent}{}
		\pre{r = r_0 \y esReunionValida(r)}
		\post{result \Iff |r|=|r_0| \yLuego (tieneLaMitadDeLasMuestras(r,r_0)  \y \\
			           posicionesImpares(r,r_0)) \\
								}
	\end{proc}
\end{addmargin}
\vspace{1cm}

\begin{addmargin}[4em]{0em}
	\auxpred{tieneLaMitadDeLasMuestras}{\In r_0 :reunion, \In r :reunion}{\\
		(\forall i :\ent)(\forall j :\ent)(0 \leq i < |r_0| \y 0 \leq j < |r|) \implicaLuego 
		\left \lfloor{\frac{|(r_0[j])_0|}{2}} \right \rfloor = |(r[i])_0| \\
	}
\end{addmargin}
\vspace{1cm}

\begin{addmargin}[4em]{0em}
	\auxpred{posicionesImpares}{\In r_0 :reunion, \In r :reunion}{\\
		(\forall i :\ent)(0 \leq i < |r_0|) \implicaLuego tomaLasPosicionesImpares((r_0[i])_0,(r[i])_0) \\
	}
\end{addmargin}
\vspace{1cm}

\begin{addmargin}[4em]{0em}
	\auxpred{tomaLasPosicionesImpares}{\In l :\TLista{\ent}, \In m :\TLista{\ent}}{\\
		(\forall i :\ent)(1 \leq i \leq |m|) \implicaLuego l[2i-1] = m[i-1] \\
	}
\end{addmargin}
\vspace{1cm}


\newpage
{\Large Ejercicio 5\par}
\vspace{0.5cm}

\begin{addmargin}[4em]{0em}
	\begin{proc}{relentizar}{\Inout r :reunion, \In prof:\ent, \In freq:\ent}{}
		\pre{r = r_0}
		\post{result \Iff |r|=|r_0| \yLuego (tieneElDobleDeLasMuestras(r,r_0)  \y \\
			           lasMuestrasExtrasSonElPromedioDeLosPuntosVecinos(r,r_0)) \y \\
								 lasMuestrasNoExtrasSonLasOriginales(r,r_0)) \\
								}
	\end{proc}
\end{addmargin}
\vspace{1cm}

\begin{addmargin}[4em]{0em}
	\auxpred{tieneElDobleDeLasMuestras}{\In r_0 :reunion, \In r :reunion}{\\
		(\forall i :\ent)(\forall j :\ent)(0 \leq i < |r_0| \y 0 \leq j < |r|) \implicaLuego |(r_0[i])_0| = 2|(r[j])_0|-1 \\
	}
\end{addmargin}
\vspace{1cm}

\begin{addmargin}[4em]{0em}
	\auxpred{lasMuestrasExtrasSonElPromedioDeLosPuntosVecinos}{\In r_0 :reunion, \In r :reunion}{\\
		(\forall i :\ent)(0 \leq i < |r_0|) \implicaLuego lasMuestrasImparesSonElPromedio((r_0[i])_0,(r[i])_0) \\
	}
\end{addmargin}
\vspace{1cm}

\begin{addmargin}[4em]{0em}
	\auxpred{lasMuestrasImparesSonElPromedio}{\In l :\TLista{\ent}, \In m :\TLista{\ent}}{\\
		(\forall i :\ent)(0 \leq i < |l|-1) \implicaLuego \left \lfloor{\frac{l[i]+l[i+1]}{2}} \right \rfloor = m[2i+1] \\
	}
\end{addmargin}
\vspace{1cm}

\begin{addmargin}[4em]{0em}
	\auxpred{lasMuestrasNoExtrasSonLasOriginales}{\In r_0 :reunion, \In r :reunion}{\\
		(\forall i :\ent)(0 \leq i < |r_0|) \implicaLuego lasMuestrasParesSonLasOriginales((r_0[i])_0,(r[i])_0) \\
	}
\end{addmargin}
\vspace{1cm}

\begin{addmargin}[4em]{0em}
	\auxpred{lasMuestrasParesSonLasOriginales}{\In l :\TLista{\ent}, \In m :\TLista{\ent}}{\\
		(\forall i :\ent)(0 \leq i < |l|) \implicaLuego l[i] = m[2i] \\
}
\end{addmargin}
\vspace{1cm}

\end{document}






