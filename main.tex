\documentclass{article}
\input{Algo1Macros}
\usepackage{caratula}
\usepackage{scrextend}


\begin{document}

%Carátula
\titulo{TP 1 - Reuniones Remotas}
\subtitulo{Grupo 3}
\fecha{22 de Mayo de 2020}
\materia{Algoritmos y Estructuras de Datos 1}
\integrante{González Narvarte, Francisco}{519/15}{francisco13\_95@live.com}
\integrante{Giménez, Iván Manuel}{374/18}{ivangimenez8727@gmail.com}
\integrante{Demare, Matías Nicolás}{762/19}{matiasdemare@gmail.com}
\integrante{Cubino, Santiago}{829/19}{sscubino@gmail.com}
\maketitle

%Creación de índice
\tableofcontents
\newpage

% End carátula


\addcontentsline{toc}{section}{Notas}
\section*{Notas}
\vspace{0.5cm}
\subsection*{Pertenece ($\in$)}
En algunos ejercicios usamos el reemplazo sintáctico '$elem \in s$' como equivalente
al siguiente predicado '$pertenece(elem, s)$' para disminuir la verbosidad de los predicados:
\vspace{0.5cm}
\begin{addmargin}[4em]{0em}
	\auxpred{pertenece}{elem :T, s :\TLista{T}}{\\
		(\exists i:\ent)(0 \leq i < |s| \yLuego s[i]=elem)\\
	}
\end{addmargin}

\vspace{0.5cm}
\subsection*{Tipos de datos}
Usamos los siguientes reemplazos en los tipos de datos:
\begin{itemize}
	\item $se\tilde{n}al=\TLista{\ent}$
	\item $reunion=\TLista{\tuple{se\tilde{n}al}{\ent}}$
	\item $intervalo=\tuple{\ent}{\ent}$
\end{itemize}
\newpage



\addcontentsline{toc}{section}{Ejercicio 1}
\section*{Ejercicio 1}
\vspace{1cm}
\begin{addmargin}[4em]{0em}
	\begin{proc}{esSeñal}{\In s :\TLista{\ent}, \In prof:\ent, \In freq:\ent, \Out b :\bool}{}
		\pre{\True}
		\post{b \Iff esSe\tilde{n}alValida(s,prof,freq)}
	\end{proc}
\end{addmargin}
\vspace{0.5cm}

\begin{addmargin}[4em]{0em}
	\auxpred{esSe\tilde{n}alValida}{\In s :\TLista{\ent}, \In prof:\ent, \In freq:\ent}{\\
		(freq=8 \lor freq=32) \y \\
		(prof=8 \lor prof=16 \lor prof=32) \y \\
		duracion(s, freq) > 1 \y \\
		respetaProfundidad(s, prof) \\
	}
\end{addmargin}
\vspace{0.5cm}

\begin{addmargin}[4em]{0em}
	% Devuelve duracion en segundos
	\aux{duracion}{s :\TLista{\ent}, freq:\ent}{\float}{(\IfThenElse{freq=0}{0}{\frac{|s|}{1000*freq}})}
\end{addmargin}
\vspace{0.5cm}

\begin{addmargin}[4em]{0em}
	\auxpred{respetaProfundidad}{s :\TLista{\ent}, prof :\ent}{\\
		(\forall i :\ent)(0 \leq i < |s|) \implicaLuego (-2^{prof-1} \leq s[i] \leq 2^{prof-1}-1)\\
	}
\end{addmargin}
\newpage



\addcontentsline{toc}{section}{Ejercicio 2}
\section*{Ejercicio 2}
\vspace{1cm}
\begin{addmargin}[4em]{0em}
	\begin{proc}{seEnojó?}{\In señal :\TLista{\ent}, \In umbral:\ent, \In prof:\ent, \In freq:\ent, \Out b :\bool}{}
		\pre{esSe\tilde{n}alValida(se\tilde{n}al,prof,freq) \y -2^{prof-1} \leq umbral \leq 2^{prof-1}-1)}
		\post{ \\
			b \Iff (\exists subse\tilde{n}al :\TLista{\ent})( \\
				incluida(se\tilde{n}al, subse\tilde{n}al) \y \\
				duracion(subse\tilde{n}al, freq) > 5 \y \\
				superaUmbral(subse\tilde{n}al, umbral)
			) \\
		}
	\end{proc}
\end{addmargin}
\vspace{0.5cm}

\begin{addmargin}[4em]{0em}
	\auxpred{incluida}{s :\TLista{\ent}, subseq :\TLista{\ent}}{\\
		(\exists i :\ent)(0 \leq i \leq |s|-|subseq|) \yLuego \\
		(\forall j :\ent)(0 \leq j < |subseq| \implicaLuego subseq[j]=s[j+i]) \\
	}
\end{addmargin}
\vspace{0.5cm}

\begin{addmargin}[4em]{0em}
	% Devuelve duracion en segundos (freq no es cero por la Pre)
	\aux{duracion}{s :\TLista{\ent}, freq:\ent}{\float}{\frac{|s|}{1000*freq}}
\end{addmargin}
\vspace{0.5cm}

\begin{addmargin}[4em]{0em}
	\auxpred{superaUmbral}{s :\TLista{\ent}, umbral :\ent}{\\
		(\forall i :\ent)(0 \leq i < |s|) \implicaLuego s[i]>umbral) \\
	}
\end{addmargin}
\newpage



\addcontentsline{toc}{section}{Ejercicio 3}
\section*{Ejercicio 3}
\vspace{1cm}
\begin{addmargin}[4em]{0em}
	\begin{proc}{esReunionValida}{\In r :reunion, \In prof:\ent, \In freq:\ent, \Out result :\bool}{}
		\pre{\True}
		\post{result \Iff esReunion(r, prof, freq)}
	\end{proc}
\end{addmargin}
\vspace{0.5cm}

\begin{addmargin}[4em]{0em}
	\auxpred{esReunion}{\In r :reunion, \In prof:\ent, \In freq:\ent}{\\
		|r|>0 \yLuego todasSonSe\tilde{n}ales(r) \y\\
		muestrasDeIgualTama\tilde{n}o(r)  \y \\
		unaSe\tilde{n}alPorPersona(r) \\
		}
\end{addmargin}
\vspace{0.5cm}

\begin{addmargin}[4em]{0em}
	\auxpred{todasSonSe\tilde{n}ales}{\In r :reunion}{
		(\forall i:\ent)(0 \leq i < |r| \implicaLuego esSe\tilde{n}alValida(r[i]_{0})
		}
\end{addmargin}
\vspace{0.5cm}

\begin{addmargin}[4em]{0em}
	\auxpred{muestrasDeIgualTama\tilde{n}o}{\In r :reunion}{\\
		(\forall i :\ent)(0 \leq i < |r|) \implicaLuego |(r[0])_0| = |(r[i])_0| \\
	}
\end{addmargin}
\vspace{0.5cm}

\begin{addmargin}[4em]{0em}
	\auxpred{unaSe\tilde{n}alPorPersona}{\In r :reunion}{\\
		(\forall i :\ent)((\forall j :\ent)((0 \leq i < |r|) \y (0\leq j < |r|) \y (i \neq j) \implicaLuego(\\
		(r[i])_1 \neq (r[j])_1 \y 0\leq (r[i])_1<|r|)))\\
	}
\end{addmargin}
\newpage



\addcontentsline{toc}{section}{Ejercicio 4}
\section*{Ejercicio 4}
\vspace{1cm}
\begin{addmargin}[4em]{0em}
	\begin{proc}{acelerar}{\Inout r :reunion, \In prof:\ent, \In freq:\ent}{}
		\pre{r = r_0}
		\post{result \Iff (tieneLaMitadDeLasMuestras(r,r_0)  \y \\
			           posicionesImpares(r,r_0)) \\
			}
	\end{proc}
\end{addmargin}
\vspace{0.5cm}

\begin{addmargin}[4em]{0em}
	\auxpred{tieneLaMitadDeLasMuestras}{\In r_0 :reunion, \In r :reunion}{\\
		(\forall i :\ent)(\forall j :\ent)(0 \leq i < |r_0| \y 0 \leq j < |r|) \implicaLuego |(r_0[i])_0| = \frac{|(r[j])_0|}{2} \\
	}
\end{addmargin}
\vspace{0.5cm}

\begin{addmargin}[4em]{0em}
	\auxpred{posicionesImpares}{\In r_0 :reunion, \In r :reunion}{\\
		(\forall i :\ent)(\forall j :\ent)(0 \leq i < |r_0| \y 0 \leq j < |r|) \implicaLuego aux((r_0[i])_0,(r[j])_0) \\
	}
\end{addmargin}
\vspace{0.5cm}

\begin{addmargin}[4em]{0em}
	\auxpred{aux}{\In l :\TLista{\ent}, \In m :\TLista{\ent}}{\\
		(\forall i :\ent)(\forall j :\ent)(0 \leq i < |l| \y 0 \leq j < |m| \y \neg esPar(i)) \implicaLuego l[i] = m[j] \\
	}
\end{addmargin}
\vspace{0.5cm}

\begin{addmargin}[4em]{0em}
	\auxpred{esPar}{\In n:\ent}{\\
		n mod 2 = 0 \\
	}
\end{addmargin}
\newpage



\addcontentsline{toc}{section}{Ejercicio 5}
\section*{Ejercicio 5}
\vspace{0.5cm}
\begin{addmargin}[4em]{0em}
	\begin{proc}{relentizar}{\Inout r :reunion, \In prof:\ent, \In freq:\ent}{}
		\pre{r = r_0}
		\post{result \Iff |r|=|r_0| \yLuego (tieneElDobleDeLasMuestras(r,r_0)  \y \\
			           lasMuestrasExtrasSonElPromedioDeLosPuntosVecinos(r,r_0)) \y \\
								 lasMuestrasNoExtrasSonLasOriginales(r,r_0)) \\
								}
	\end{proc}
\end{addmargin}
\vspace{1cm}

\begin{addmargin}[4em]{0em}
	\auxpred{tieneElDobleDeLasMuestras}{\In r_0 :reunion, \In r :reunion}{\\
		(\forall i :\ent)(\forall j :\ent)(0 \leq i < |r_0| \y 0 \leq j < |r|) \implicaLuego |(r_0[i])_0| = 2|(r[j])_0|-1 \\
	}
\end{addmargin}
\vspace{1cm}

\begin{addmargin}[4em]{0em}
	\auxpred{lasMuestrasExtrasSonElPromedioDeLosPuntosVecinos}{\In r_0 :reunion, \In r :reunion}{\\
		(\forall i :\ent)(0 \leq i < |r_0|) \implicaLuego lasMuestrasImparesSonElPromedio((r_0[i])_0,(r[i])_0) \\
	}
\end{addmargin}
\vspace{1cm}

\begin{addmargin}[4em]{0em}
	\auxpred{lasMuestrasImparesSonElPromedio}{\In l :\TLista{\ent}, \In m :\TLista{\ent}}{\\
		(\forall i :\ent)(0 \leq i < |l|-1) \implicaLuego \left \lfloor{\frac{l[i]+l[i+1]}{2}} \right \rfloor = m[2i+1] \\
	}
\end{addmargin}
\vspace{1cm}

\begin{addmargin}[4em]{0em}
	\auxpred{lasMuestrasNoExtrasSonLasOriginales}{\In r_0 :reunion, \In r :reunion}{\\
		(\forall i :\ent)(0 \leq i < |r_0|) \implicaLuego lasMuestrasParesSonLasOriginales((r_0[i])_0,(r[i])_0) \\
	}
\end{addmargin}
\vspace{1cm}

\begin{addmargin}[4em]{0em}
	\auxpred{lasMuestrasParesSonLasOriginales}{\In l :\TLista{\ent}, \In m :\TLista{\ent}}{\\
		(\forall i :\ent)(0 \leq i < |l|) \implicaLuego l[i] = m[2i] \\
}
\end{addmargin}
\vspace{1cm}
\newpage



\addcontentsline{toc}{section}{Ejercicio 6}
\section*{Ejercicio 6}
\vspace{1cm}
\begin{addmargin}[4em]{0em}
	\begin{proc}{tonosDeVozElevados}{\In r : reunión, \In prof:\ent, \In freq:\ent, \Out hablantes :\TLista{hablante}}{}
		\pre{\\
			esReunion(r, prof, frec)\\
			}
		\post{ \\
			(|r| > 0 \implica |hablantes| > 0) \yLuego \\
			(\forall i :\ent)( i \in hablantes \iff (0 \leq i < |r| \yLuego tieneElTonoMasAlto(i, r)) \\
		}
	\end{proc}

\end{addmargin}
\vspace{0.5cm}


\begin{addmargin}[4em]{0em}
	\auxpred{tieneElTonoMasAlto}{i :\ent, r : reunion}{ \\
		(\exists j :\ent) (0 \leq j < |r| \yLuego r[j]_{1}=i \y (\\
		(\forall k :\ent) (0 \leq k < |r| \implicaLuego tono(r[j]_{0}) \geq tono(r[k]_{0}))))\\
	}

\end{addmargin}
\vspace{0.5cm}

\begin{addmargin}[4em]{0em}
	\aux{tono}{s: señal}{\float}{\sum_{i=0}^{|s|-1} \frac{s[i]}{|s|} }

\end{addmargin}
\newpage




\addcontentsline{toc}{section}{Ejercicio 7}
\section*{Ejercicio 7}
\vspace{1cm}
\begin{addmargin}[4em]{0em}
	\begin{proc}{ordenar}{\Inout r:reunion, \In prof:\ent, \In freq:\ent}{}
		\pre{r=r_{0} \y esReunion(r_{0})}
		\post{estanOrdenados(r) \y sonLaMismaReunion(r_{0},r)}
	\end{proc}
\end{addmargin}
\vspace{0.5cm}

\begin{addmargin}[4em]{0em}
    \auxpred{estanOrdenados}{r:reunion}{\\
    (\forall i:\ent)(0\leq i<|r|-1)\implicaLuego tono(r[i]{0})>tono(r[i+1]{0})\\
    }
\end{addmargin}
\vspace{0.5cm}

\begin{addmargin}[4em]{0em}
    \auxpred{sonLaMismaReunion}{r_{0}:reunion, r: reunion}{\\
	(\forall s: se\tilde{n}al)(s \in r_{0} \iff s \in r)\\
    }
\end{addmargin}
\newpage



\addcontentsline{toc}{section}{Ejercicio 8}
\section*{Ejercicio 8}
\vspace{0.5cm}
\begin{addmargin}[4em]{0em}
	\begin{proc}{silencios}{\In s : señal, \In prof:\ent, \In freq:\ent, \In umbral:\ent, \Out intervalos :\TLista{intervalo}}{}
		\pre{\\
			duracion(se\tilde{n}al, freq) > 0 \y\\
			respetaProfundidad(se\tilde{n}al, prof) \y\\
			-2^{prof-1} \leq umbral \leq 2^{prof-1}-1\\
		}
		\post{ \\
			(\forall interv:intervalo)((interv \in intervalos) \Iff (duracionDeIntervalo(interv, freq) > 0.1 \yLuego\\
			tieneLimitesValidos(interv, s) \yLuego\\
			esIntervaloSilencioso(interv, s, ubmral)))\y \\
			noHayRepetidos(intervalos) \\
		}
	\end{proc}
\end{addmargin}
\vspace{0.5cm}

\begin{addmargin}[4em]{0em}
	% Devuelve duracion en segundos
	\aux{duracion}{s :\TLista{\ent}, freq:\ent}{\float}{(\IfThenElse{freq=0}{0}{\frac{|s|}{1000*freq}})}
\end{addmargin}
\vspace{1cm}

\begin{addmargin}[4em]{0em}
	\auxpred{respetaProfundidad}{s :\TLista{\ent}, prof :\ent}{\\
		(\forall i :\ent)(0 \leq i < |s| \implicaLuego -2^{prof-1} \leq s[i] \leq 2^{prof-1}-1) \\
	}
\end{addmargin}
\vspace{1cm}

\begin{addmargin}[4em]{0em}
	% Devuelve duracion en segundos
	\aux{duracionDeIntervalo}{interv :intervalo, freq:\ent}{\float}{(\IfThenElse{freq=0}{0}{\frac{interv_1-interv_0}{1000*freq}})}
\end{addmargin}
\vspace{1cm}

\begin{addmargin}[4em]{0em}
	\auxpred{tieneLimitesValidos}{interv :intervalo, s : se\tilde{n}al}{\\
		0 \leq interv_0 \y interv_0 \leq interv_1 \y interv_1 < |s| \\
	}
\end{addmargin}
\vspace{1cm}

\begin{addmargin}[4em]{0em}
	\auxpred{esIntervaloSilencioso}{interv :intervalo,  s : se\tilde{n}al, umbral:\ent}{\\
		(\forall i :\ent)(interv_1 \leq i \leq interv_2 \implicaLuego (|s[i]| \leq umbral)) \y \\
		(interv_0=0 \oLuego |s[interv_0-1]| > umbral) \y (interv_1=|s|-1 \oLuego |s[interv_1+1]| > umbral)\\
	}
\end{addmargin}
\vspace{1cm}

\begin{addmargin}[4em]{0em}
	\auxpred{noHayRepetidos}{interv :\TLista{intervalo}}{\\
		(\forall i :\ent)((\forall j:\ent)((i\neq j \y 0\leq i <|interv| \y 0\leq j <|interv|) \implicaLuego (interv[i]_0 \neq interv[j]_0 \lor interv[i]_1 \neq interv[j]_1))) \\
	}
\end{addmargin}
\vspace{1cm}
\newpage



\addcontentsline{toc}{section}{Ejercicio 9}
\section*{Ejercicio 9}
\vspace{0.5cm}
\begin{addmargin}[4em]{0em}
	\begin{proc}{hablantesSuperpuestos}{\In r :reunion, \In prof:\ent, \In freq:\ent, \In umbral:\ent, \Out result :\bool}{}
		\pre{\\
			-2^{prof-1} \leq umbral \leq 2^{prof-1}-1\\
		}
		\post{ \\
			result \Iff ((\exists interv:intervalo)(duracionDeIntervalo(interv, freq) > 0.1 \yLuego\\
			((\exists s_1 : se\tilde{n}al)(\exists s_2 : se\tilde{n}al)(se\tilde{n}alEnReunion(s_1, r) \y (se\tilde{n}alEnReunion(s_2, r)) \yLuego \\
			(tieneLimitesValidos(interv, s_1) \y tieneLimitesValidos(interv, s_2)) \yLuego\\
			(esIntervaloRuidoso(interv, s_1, umbral) \y  esIntervaloRuidoso(interv, s_2, umbral))))) \\
		}
	\end{proc}
\end{addmargin}
\vspace{0.5cm}

\begin{addmargin}[4em]{0em}
	% Devuelve duracion en segundos
	\aux{duracionDeIntervalo}{\In interv :intervalo,\In freq:\ent}{\float}{(\IfThenElse{freq=0}{0}{\frac{interv_1-interv_0}{1000*freq}})}
\end{addmargin}
\vspace{1cm}

\begin{addmargin}[4em]{0em}
	\auxpred{se\tilde{n}alEnReunion}{\In s : se\tilde{n}al,\In r : reunion}{\\
		(\exists i :\ent)(0 \leq i < |r| \y |s| = |(reunion[i])0|) \yLuego (\forall j :\ent)(0 \leq j < |s| \implicaLuego s[j] = (reunion[i])_0[j]) \\
	}
\end{addmargin}
\vspace{1cm}

\begin{addmargin}[4em]{0em}
	\auxpred{tieneLimitesValidos}{\In interv :intervalo,\In s : se\tilde{n}al}{\\
		0 \leq interv_0 \y interv_0 \leq interv_1 \y interv_1 < |s| \\
	}
\end{addmargin}
\vspace{1cm}

\begin{addmargin}[4em]{0em}
	\auxpred{esIntervaloRuidoso}{\In interv :intervalo,\In  s : se\tilde{n}al,\In umbral:\ent}{\\
		(\forall i :\ent)(interv_1 \leq i \leq interv_2 \implicaLuego (|s[i]| > umbral)) \\
	}
\end{addmargin}
\vspace{1cm}
\newpage



\addcontentsline{toc}{section}{Ejercicio 10}
\section*{Ejercicio 10}
\vspace{1cm}
\begin{addmargin}[4em]{0em}
	\begin{proc}{reconstruir}{\In s: señal, \In frec: \ent, \In prof:\ent, \Out result: señal}{}
		\pre{esSe\tilde{n}al(s, prof, frec) \y tieneAlMenosDosValoresNoNulos(s)}
		\post{ \\
			|s|=|result| \yLuego (\\
			(\forall i : \ent)(0\leq i<|s|)\implicaLuego (\\
			(s[i] = result[i] \y s[i]\neq 0) \vee seReparo(i,s,result)))\\
		}
	\end{proc}
\end{addmargin}
\vspace{0.5cm}

\begin{addmargin}[4em]{0em}
	\auxpred{seReparo}{\In i: \ent, \In s :se\tilde{n}al, \In result :se\tilde{n}al}{\\
		(\exists a:\ent)(\exists b:\ent)(0 \leq a < b < |s| \yLuego sonLosMasCercanos(a,b,i,s) \yLuego\\
		result[i] = \left \lfloor{\frac{a + b}{2}} \right \rfloor)\\
	}
\end{addmargin}
\vspace{0.5cm}

\begin{addmargin}[4em]{0em}
	\auxpred{sonLosMasCercanos}{\In a: \ent, \In b :\ent, \In i :\ent, \In s: se\tilde{n}al}{\\
		(\forall k : \ent)((0 \leq k < |s| \yLuego (|k-i|<|a-i| \o |k-i|<|b-i|)) \implica s[k]=0)\\
	}
\end{addmargin}
\vspace{0.5cm}

\begin{addmargin}[4em]{0em}
	\auxpred{tieneAlMenosDosValoresNoNulos}{\In s: se\tilde{n}al}{\\
		(\exists a : \ent)((\exists b : \ent)((0 \leq a < |s| \y 0 \leq b < |s| \y a\neq b) \yLuego\\
		s[a]\neq 0 \y s[b]\neq 0))\\
	}
\end{addmargin}


\end{document}





				